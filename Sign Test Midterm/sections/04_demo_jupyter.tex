\subsection{Setup Demo}

\begin{frame}[fragile]{Demo Setup - Ví dụ Thuốc giảm cân}
    \footnotesize
    \textbf{Bài toán:}
    \begin{itemize}\itemsep0pt
        \item 12 người tham gia thử nghiệm thuốc giảm cân
        \item Đo cân nặng trước và sau 2 tháng sử dụng
        \item Câu hỏi: Thuốc có hiệu quả không?
    \end{itemize}
    \vspace{0.15cm}
    \textbf{Chuẩn bị Jupyter Notebook:}
    \begin{lstlisting}[language=Python, caption=Cell 1: Import libraries]
import numpy as np
import pandas as pd
from scipy.stats import binomtest
import matplotlib.pyplot as plt

# Du lieu: Can nang 12 nguoi
data = {
    'ID': range(1, 13),
    'Truoc_kg': [85, 90, 78, 92, 88, 76, 95, 82, 79, 91, 87, 84],
    'Sau_kg':   [82, 88, 79, 90, 85, 77, 93, 80, 78, 89, 86, 83]
}

df = pd.DataFrame(data)
print(df.head())
    \end{lstlisting}
    \vspace{0.1cm}
    
    % ẢNH JUPYTER CELL 1
    \includegraphics[width=0.7\textwidth]{jupyter_cell1_output.png}
\end{frame}

\subsection{Tính hiệu số và dấu}

\begin{frame}[fragile]{Demo Step 1: Tính hiệu số và xác định dấu}
    \footnotesize
    \begin{lstlisting}[language=Python, caption=Cell 2: Calculate differences and signs]
# Buoc 1: Tinh hieu (truoc - sau)
df['Hieu_kg'] = df['Truoc_kg'] - df['Sau_kg']

# Buoc 2: Xac dinh dau
def assign_sign(x):
    if x > 0: return '+'
    elif x < 0: return '-'
    else: return '0'

df['Dau'] = df['Hieu_kg'].apply(assign_sign)
df_filtered = df[df['Hieu_kg'] != 0]

# Hien thi ket qua
print(df_filtered[['ID', 'Truoc_kg', 'Sau_kg', 'Hieu_kg', 'Dau']])
    \end{lstlisting}
    \vspace{0.15cm}
    
    % ẢNH BẢNG KẾT QUẢ
    \includegraphics[width=0.75\textwidth]{jupyter_output_table.png}
\end{frame}

\subsection{Thực hiện kiểm định}

\begin{frame}[fragile]{Demo Step 2: Thống kê \& Kiểm định}
    \footnotesize
    \begin{lstlisting}[language=Python, caption=Cell 3: Perform Sign Test]
# Dem so dau +
n_plus = (df_filtered['Hieu_kg'] > 0).sum()
n_total = len(df_filtered)
print(f"So dau +: {n_plus}/{n_total}")

# Kiem dinh Binomial (one-tailed: greater)
result = binomtest(n_plus, n_total, p=0.5, alternative='greater')
print(f"P-value: {result.pvalue:.4f}")

# Ket luan
alpha = 0.05
if result.pvalue < alpha:
    print(f"BAC BO H0 (p={result.pvalue:.4f} < {alpha})")
    print("=> CO BANG CHUNG: Thuoc co hieu qua!")
else:
    print(f"GIU H0")
    \end{lstlisting}
    \vspace{0.2cm}
    
    % ẢNH KẾT QUẢ STATS
    \includegraphics[width=0.6\textwidth]{jupyter_output_stats.png}
\end{frame}

\subsection{Trực quan hóa}

\begin{frame}[fragile]{Demo Step 3: Visualization}
    \footnotesize
    \begin{columns}[T]
        \begin{column}{0.58\textwidth}
            \begin{lstlisting}[language=Python, caption=Cell 4: Visualize results, basicstyle=\ttfamily\tiny]
fig, axes = plt.subplots(1, 2, figsize=(12, 5))

# Plot 1: Scatter truoc vs sau
ax1 = axes[0]
ax1.scatter(df['Truoc_kg'], df['Sau_kg'], s=100)
ax1.plot([75, 100], [75, 100], 'r--')
ax1.set_xlabel('Truoc (kg)')
ax1.set_ylabel('Sau (kg)')

# Plot 2: Bar chart hieu so
ax2 = axes[1]
colors = ['green' if x > 0 else 'red' for x in df['Hieu_kg']]
ax2.bar(df['ID'], df['Hieu_kg'], color=colors)
ax2.axhline(0, color='black', lw=2)

plt.tight_layout()
plt.show()
            \end{lstlisting}
        \end{column}
        \begin{column}{0.40\textwidth}
            % ẢNH BIỂU ĐỒ
            \includegraphics[width=\textwidth]{jupyter_plots.png}
            
            \vspace{0.15cm}
            \begin{alertblock}{Insight từ plots}
                \begin{itemize}\itemsep0pt
                    \item Hầu hết điểm dưới đường chéo
                    \item Đa số bars màu xanh (dương)
                    \item Chỉ 2 người tăng cân
                \end{itemize}
            \end{alertblock}
        \end{column}
    \end{columns}
\end{frame}