% =========================================
% SECTION 7: CÂU HỎI & TRẢ LỜI (FULL DETAIL + INTERACTIVE)
% =========================================

\subsection{Danh sách câu hỏi}

% --- MENU CHÍNH (MỤC LỤC CÂU HỎI) ---
% Slide này dùng để chọn câu hỏi
\begin{frame}[label=qna_menu, shrink=5]{Danh sách Câu hỏi Thảo luận}
    \vspace{0.2cm}
    \textit{Bấm vào câu hỏi để xem chi tiết:}
    \vspace{0.2cm}
    
    \begin{columns}[T]
        \begin{column}{0.48\textwidth}
            \begin{itemize}\itemsep8pt
                \item[\textbf{Q1.}] \hyperlink{q1}{\textbf{So sánh:} Sign Test vs Wilcoxon Signed-Rank?}
                \item[\textbf{Q2.}] \hyperlink{q2}{\textbf{Xử lý số liệu:} Tại sao loại bỏ Ties (0)?}
                \item[\textbf{Q3.}] \hyperlink{q3}{\textbf{Dữ liệu:} Dùng cho Likert scale được không?}
                \item[\textbf{Q4.}] \hyperlink{q4}{\textbf{Cỡ mẫu:} Vấn đề khi mẫu nhỏ (n < 10)?}
                \item[\textbf{Q5.}] \hyperlink{q5}{\textbf{Báo cáo:} Viết kết quả chuẩn academic?}
            \end{itemize}
        \end{column}
        
        \begin{column}{0.48\textwidth}
            \begin{itemize}\itemsep8pt
                \item[\textbf{Q6.}] \hyperlink{q6}{\textbf{Nâng cao:} Multiple comparisons correction?}
                \item[\textbf{Q7.}] \hyperlink{q7}{\textbf{Công cụ:} Các thư viện Python/R/SPSS?}
                \item[\textbf{Q8.}] \hyperlink{q8}{\textbf{Giả định:} Assumptions nào bắt buộc?}
                \item[\textbf{Q9.}] \hyperlink{q9}{\textbf{Hiệu năng:} Power Analysis \& Sample Size?}
                \item[\textbf{Q10.}] \hyperlink{q10}{\textbf{Mở rộng:} Các biến thể (McNemar, Friedman)?}
            \end{itemize}
        \end{column}
    \end{columns}
\end{frame}

% =========================================
% CÁC SLIDE CHI TIẾT (KHÔI PHỤC NỘI DUNG GỐC)
% =========================================

% --- Q1 ---
\begin{frame}[allowframebreaks, label=q1]{Q1: So sánh Sign Test vs Wilcoxon}
    \footnotesize
    \begin{block}{Câu hỏi}
        Sign Test khác với Wilcoxon Signed-Rank Test như thế nào? Khi nào nên dùng cái nào?
    \end{block}
    
    \vspace{0.2cm}
    \textbf{Trả lời chi tiết:}
    
    \begin{table}
        \centering
        \tiny
        \begin{tabular}{p{2.5cm}|p{3.5cm}|p{3.5cm}}
            \toprule
            \textbf{Tiêu chí} & \textbf{Sign Test} & \textbf{Wilcoxon} \\
            \midrule
            Thông tin sử dụng & Chỉ dấu (+/-) & Dấu + Độ lớn (Rank) \\
            Power & Thấp hơn (khoảng 64\% của t-test) & Cao hơn (khoảng 95\% của t-test) \\
            Assumptions & \cellcolor{green!20}Ít nhất (chỉ cần độc lập) & Thêm: phân phối đối xứng (symmetric) \\
            Dễ hiểu & \cellcolor{green!20}Rất trực quan & Phức tạp hơn về tính toán \\
            Khi nào dùng & Dữ liệu rất lệch, nhiều Outliers & Dữ liệu gần đối xứng \\
            \bottomrule
        \end{tabular}
    \end{table}
    
    \vspace{0.2cm}
    \structure{\textbf{Khuyến nghị:}}
    \begin{itemize}\itemsep0pt
        \item \textbf{Sign Test:} Khi dữ liệu rất "ồn" (noisy), nhiều giá trị ngoại lai cực đoan, hoặc chỉ quan tâm đến hướng thay đổi.
        \item \textbf{Wilcoxon:} Khi dữ liệu tương đối đối xứng và muốn đạt độ mạnh (power) cao hơn.
    \end{itemize}

    \vfill \hfill \hyperlink{qna_menu}{\beamerbutton{\icon{list} Quay lại Menu}}
\end{frame}

% --- Q2 ---
\begin{frame}[allowframebreaks, label=q2]{Q2: Xử lý Ties (Hiệu số = 0)}
    \footnotesize
    \begin{block}{Câu hỏi}
        Nếu có nhiều hiệu số bằng 0, có ảnh hưởng gì không? Có cách xử lý nào khác?
    \end{block}
    
    \vspace{0.2cm}
    \textbf{Trả lời:}
    
    \begin{columns}[T]
        \begin{column}{0.48\textwidth}
            \textbf{Ảnh hưởng của ties:}
            \begin{itemize}\itemsep0pt
                \item Giảm kích thước mẫu hiệu dụng (Effective sample size).
                \item Giảm sức mạnh thống kê (Statistical power).
                \item \alert{Cảnh báo:} Nếu > 20\% dữ liệu là ties thì kết quả có thể thiếu tin cậy.
            \end{itemize}
            
            \vspace{0.15cm}
            \textbf{Phương pháp xử lý:}
            \begin{enumerate}\itemsep0pt
                \item \structure{Standard:} Loại bỏ hoàn toàn (Khuyên dùng).
                \item Chia đều: Một nửa vào nhóm (+), một nửa vào nhóm (-).
                \item Random assign: Gán ngẫu nhiên dấu.
            \end{enumerate}
        \end{column}
        
        \begin{column}{0.48\textwidth}
            \textbf{Ví dụ minh họa:}
            
            Tổng n = 20, có 5 ties (25\% số liệu là 0)
            
            \begin{itemize}\itemsep0pt
                \item \structure{Cách 1 (Loại):} Phân tích với $n = 15$.
                \item \structure{Cách 2 (Chia):} Tính 2.5 cho (+) và 2.5 cho (-).
                \item \structure{Cách 3 (Random):} Tung xu cho 5 trường hợp đó.
            \end{itemize}
            
            \vspace{0.15cm}
            \begin{alertblock}{Lưu ý}
                Cách 1 (loại bỏ) là "standard practice" trong hầu hết các phần mềm thống kê (SPSS, R, Python).
            \end{alertblock}
        \end{column}
    \end{columns}
    \vfill \hfill \hyperlink{qna_menu}{\beamerbutton{Quay lại Menu}}
\end{frame}

% --- Q3 ---
\begin{frame}[allowframebreaks, label=q3]{Q3: Áp dụng cho data thứ bậc (Likert Scale)}
    \footnotesize
    \begin{block}{Câu hỏi}
        Sign Test có thể dùng cho dữ liệu thứ bậc (ordinal) như Likert scale (1-5) không?
    \end{block}
    
    \vspace{0.2cm}
    \textbf{Trả lời: ĐÚNG!} Sign Test cực kỳ phù hợp cho ordinal data.
    
    \vspace{0.15cm}
    \textbf{Ví dụ - Khảo sát hài lòng:}
    
    \begin{columns}[T]
        \begin{column}{0.52\textwidth}
            \textbf{Thiết kế:}
            \begin{itemize}\itemsep0pt
                \item Survey trước và sau cải tiến dịch vụ.
                \item Scale: 1 (Rất tệ) $\to$ 5 (Rất tốt).
                \item So sánh: Điểm Trước vs Điểm Sau.
            \end{itemize}
            
            \vspace{0.15cm}
            \textbf{Phân tích:}
            \begin{itemize}\itemsep0pt
                \item Sau > Trước (VD: 3 lên 4) $\to$ Dấu (+)
                \item Sau < Trước (VD: 5 xuống 2) $\to$ Dấu (-)
                \item Sau = Trước $\to$ Loại
            \end{itemize}
        \end{column}
        
        \begin{column}{0.46\textwidth}
            \textbf{Tại sao phù hợp?}
            \begin{enumerate}\itemsep0pt
                \item Likert scale là ordinal, khoảng cách giữa 1-2 chưa chắc bằng 4-5.
                \item T-test yêu cầu tính Mean (cộng trừ), điều này không chặt chẽ với Likert.
                \item Sign Test chỉ cần thứ tự (lớn hơn/nhỏ hơn), không cần đo khoảng cách.
            \end{enumerate}
        \end{column}
    \end{columns}
    \vfill \hfill \hyperlink{qna_menu}{\beamerbutton{Quay lại Menu}}
\end{frame}

% --- Q4 ---
\begin{frame}[allowframebreaks, label=q4]{Q4: Vấn đề với Sample size nhỏ}
    \footnotesize
    \begin{block}{Câu hỏi}
        Với n = 6 hoặc n = 7, Sign Test có đáng tin không? Cần làm gì?
    \end{block}
    
    \vspace{0.2cm}
    \textbf{Trả lời:}
    
    \begin{columns}[T]
        \begin{column}{0.48\textwidth}
            \textbf{Vấn đề "Bước nhảy" P-value:}
            \begin{itemize}\itemsep0pt
                \item Do phân phối nhị thức là rời rạc.
                \item P-values sẽ "nhảy" qua các mốc, khó đạt chính xác 0.05.
            \end{itemize}
            
            \vspace{0.15cm}
            \textbf{Ví dụ thực tế với n=6:}
            \begin{itemize}\itemsep0pt
                \item Nếu 5 dấu (+): p = 0.219 (Quá lớn)
                \item Nếu 6 dấu (+): p = 0.031 (Nhỏ hơn 0.05)
                \item Không có p-value nào nằm giữa 0.031 và 0.219!
            \end{itemize}
        \end{column}
        
        \begin{column}{0.48\textwidth}
            \textbf{Giải pháp:}
            \begin{enumerate}\itemsep0pt
                \item \structure{Exact test:} Luôn dùng Binomial Exact (không dùng xấp xỉ chuẩn).
                \item \structure{Report exact p:} Báo cáo p-value chính xác thay vì chỉ nói < 0.05.
                \item \structure{Power:} Chấp nhận rằng power sẽ rất thấp.
                \item \structure{Data:} Cố gắng thu thập thêm mẫu nếu có thể (ít nhất n=10).
            \end{enumerate}
        \end{column}
    \end{columns}
    \vfill \hfill \hyperlink{qna_menu}{\beamerbutton{Quay lại Menu}}
\end{frame}

% --- Q5 ---
\begin{frame}[allowframebreaks, label=q5]{Q5: Báo cáo kết quả (Reporting)}
    \footnotesize
    \begin{block}{Câu hỏi}
        Khi viết báo cáo khoa học hoặc luận văn, cần trình bày những thông số gì?
    \end{block}
    
    \vspace{0.2cm}
    \textbf{Checklist thông tin cần thiết:}
    
    \begin{enumerate}\itemsep0pt
        \item \textbf{Sample size:}
        \begin{itemize}\itemsep0pt
            \item Tổng số quan sát ban đầu ($N$).
            \item Số ties bị loại bỏ.
            \item Kích thước mẫu hiệu dụng ($n$).
        \end{itemize}
        
        \item \textbf{Test statistic:}
        \begin{itemize}\itemsep0pt
            \item Số lượng dấu dương ($S^+$) và dấu âm ($S^-$).
            \item Tỷ lệ phần trăm (VD: 80\% cải thiện).
        \end{itemize}
        
        \item \textbf{P-value:}
        \begin{itemize}\itemsep0pt
            \item Giá trị chính xác (VD: p = 0.033).
            \item Ghi rõ là One-tailed (1 phía) hay Two-tailed (2 phía).
        \end{itemize}
    \end{enumerate}
    
    \vspace{0.2cm}
    \textbf{Ví dụ câu viết mẫu (Standard Reporting):}
    \begin{quote}
        \scriptsize
        "A Sign test was conducted to evaluate the effect of the new treatment. Of 12 participants, 1 showed no change and was excluded (effective n=11). Nine participants (81.8\%) showed improvement, compared to 2 (18.2\%) who worsened. The exact one-tailed Sign test indicated a significant improvement (p = 0.033), suggesting the treatment is effective."
    \end{quote}
    \vfill \hfill \hyperlink{qna_menu}{\beamerbutton{Quay lại Menu}}
\end{frame}

% --- Q6 ---
\begin{frame}[allowframebreaks, label=q6]{Q6: Multiple Comparisons}
    \footnotesize
    \begin{block}{Câu hỏi}
        Nếu thực hiện nhiều Sign Tests (ví dụ: test 5 loại thuốc khác nhau trên cùng 1 nhóm), có cần điều chỉnh gì không?
    \end{block}
    
    \vspace{0.2cm}
    \textbf{Trả lời: CẦN!} Đây là vấn đề kiểm định đa giả thuyết.
    
    \vspace{0.15cm}
    \begin{columns}[T]
        \begin{column}{0.48\textwidth}
            \textbf{Vấn đề:}
            \begin{itemize}\itemsep0pt
                \item Khi làm $k$ tests độc lập với $\alpha = 0.05$.
                \item Xác suất mắc ít nhất 1 lỗi loại 1 (False Positive) tăng lên chóng mặt.
                \item Công thức: $P(\text{Error}) = 1 - 0.95^k$.
                \item Với 5 tests: $1 - 0.95^5 \approx 22.6\%$ (Rất cao!).
            \end{itemize}
        \end{column}
        
        \begin{column}{0.48\textwidth}
            \textbf{Giải pháp Correction:}
            \begin{enumerate}\itemsep0pt
                \item \structure{Bonferroni:} Chia nhỏ $\alpha$.
                $$ \alpha_{new} = \frac{0.05}{5} = 0.01 $$
                (Khá bảo thủ, khó đạt ý nghĩa thống kê).
                
                \item \structure{Holm-Bonferroni:} Mạnh hơn Bonferroni, điều chỉnh theo từng bước (Sequential).
                
                \item \structure{FDR (Benjamini-Hochberg):} Kiểm soát tỷ lệ phát hiện sai, thường dùng trong Gen/Y sinh.
            \end{enumerate}
        \end{column}
    \end{columns}
    \vfill \hfill \hyperlink{qna_menu}{\beamerbutton{Quay lại Menu}}
\end{frame}

% --- Q7 ---
\begin{frame}[allowframebreaks, label=q7]{Q7: Công cụ thực hiện (Software Implementation)}
    \footnotesize
    \begin{block}{Câu hỏi}
        Có thư viện nào khác ngoài scipy.stats để làm Sign Test không? R và SPSS thì sao?
    \end{block}
    
    \vspace{0.2cm}
    \textbf{Tổng hợp các công cụ phổ biến:}
    
    \begin{columns}[T]
        \begin{column}{0.48\textwidth}
            \textbf{Python:}
            \begin{enumerate}\itemsep0pt
                \item \texttt{scipy.stats.binomtest} (\alert{Khuyến nghị nhất} - Chuẩn xác).
                \item \texttt{statsmodels.stats.descriptivestats.sign\_test} (Trả về statistic và p-value).
                \item \texttt{pingouin.sign\_test} (Giao diện thân thiện kiểu Pandas).
            \end{enumerate}
            
            \vspace{0.15cm}
            \textbf{R Language:}
            \begin{enumerate}\itemsep0pt
                \item \texttt{binom.test()} (Base R).
                \item \texttt{SIGN.test()} (Gói BSDA).
                \item \texttt{SignTest()} (Gói DescTools).
            \end{enumerate}
        \end{column}
        
        \begin{column}{0.48\textwidth}
            \textbf{Phần mềm thống kê (GUI):}
            
            \begin{itemize}\itemsep0pt
                \item \textbf{SPSS:} Menu: Analyze $\to$ Nonparametric Tests $\to$ Legacy Dialogs $\to$ 2 Related Samples $\to$ Tick chọn "Sign".
                \item \textbf{SAS:} Dùng \texttt{PROC FREQ} với tùy chọn binomial.
                \item \textbf{Stata:} Dùng lệnh \texttt{signtest var1 = var2}.
            \end{itemize}
            
            \vspace{0.15cm}
            \begin{alertblock}{Tip}
                Dù dùng công cụ nào, hãy đảm bảo bạn đang xem \textbf{Exact P-value} chứ không phải Asymptotic (xấp xỉ).
            \end{alertblock}
        \end{column}
    \end{columns}
    \vfill \hfill \hyperlink{qna_menu}{\beamerbutton{Quay lại Menu}}
\end{frame}

% --- Q8 ---
\begin{frame}[allowframebreaks, label=q8]{Q8: Kiểm tra Giả định (Assumptions)}
    \footnotesize
    \begin{block}{Câu hỏi}
        Sign Test là phi tham số (nonparametric) nhưng có assumptions nào cần check không?
    \end{block}
    
    \vspace{0.2cm}
    \textbf{Các giả định BẮT BUỘC:}
    
    \begin{enumerate}\itemsep0pt
        \item \textbf{Independence (Quan trọng nhất):}
        \begin{itemize}\itemsep0pt
            \item Các cặp quan sát phải độc lập với nhau.
            \item Ví dụ: Không thể đo 1 người 10 lần rồi coi là 10 mẫu độc lập.
            \item Nếu vi phạm $\to$ Test hoàn toàn sai.
        \end{itemize}
        
        \item \textbf{Random Sampling:}
        \begin{itemize}\itemsep0pt
            \item Mẫu phải đại diện cho tổng thể (không bị bias).
        \end{itemize}
        
        \item \textbf{Dữ liệu so sánh được:}
        \begin{itemize}\itemsep0pt
            \item Biến số phải là Continuous (Liên tục) hoặc Ordinal (Thứ bậc).
            \item Không dùng cho dữ liệu Định danh (Nominal) chưa có thứ tự.
        \end{itemize}
    \end{enumerate}
    
    \vspace{0.2cm}
    \textbf{Những điều KHÔNG cần (Ưu điểm):}
    \begin{itemize}\itemsep0pt
        \item \checkmark KHÔNG cần phân phối chuẩn (Normality).
        \item \checkmark KHÔNG cần phương sai đồng nhất (Homogeneity of variance).
        \item \checkmark KHÔNG cần mẫu lớn.
    \end{itemize}
    \vfill \hfill \hyperlink{qna_menu}{\beamerbutton{Quay lại Menu}}
\end{frame}

% --- Q9 ---
\begin{frame}[allowframebreaks, label=q9]{Q9: Power Analysis \& Sample Size}
    \footnotesize
    \begin{block}{Câu hỏi}
        Làm sao tính sample size cần thiết cho Sign Test? Power là bao nhiêu?
    \end{block}
    
    \vspace{0.2cm}
    \textbf{Power phụ thuộc vào 3 yếu tố:}
    \begin{enumerate}\itemsep0pt
        \item Sample size ($n$).
        \item Effect size (Tỷ lệ thực sự của dấu + trong quần thể, gọi là $P_{real}$).
        \item Mức ý nghĩa $\alpha$ (thường là 0.05).
    \end{enumerate}
    
    \vspace{0.15cm}
    
    \begin{columns}[T]
        \begin{column}{0.48\textwidth}
            \textbf{Ví dụ tính toán:}
            \begin{itemize}\itemsep0pt
                \item Giả thuyết $H_0: P = 0.5$.
                \item Thực tế thuốc tốt, $P_{real} = 0.7$ (70\% người dùng sẽ đỡ).
                \item Muốn Power = 0.80 (80\% khả năng phát hiện ra).
            \end{itemize}
            
            $\to$ Kết quả tính toán: Cần \alert{$n \approx 28$}.
            
            Nếu thuốc cực tốt ($P_{real} = 0.8$) $\to$ Chỉ cần \alert{$n \approx 14$}.
        \end{column}
        
        \begin{column}{0.48\textwidth}
            \textbf{Công cụ tính Power:}
            \begin{itemize}\itemsep0pt
                \item \textbf{Python:} \texttt{statsmodels.stats.power} (tùy biến class Binomial).
                \item \textbf{R:} Gói \texttt{pwr} hàm \texttt{power.binom.test()}.
                \item \textbf{G*Power:} Phần mềm miễn phí phổ biến nhất cho sinh viên/nghiên cứu sinh.
            \end{itemize}
        \end{column}
    \end{columns}
    
    \vspace{0.15cm}
    \begin{alertblock}{Ghi nhớ}
        Nếu dữ liệu tuân theo phân phối chuẩn, Sign Test yếu hơn t-test. Nhưng nếu dữ liệu lệch, Sign Test có thể mạnh hơn!
    \end{alertblock}
    \vfill \hfill \hyperlink{qna_menu}{\beamerbutton{Quay lại Menu}}
\end{frame}

% --- Q10 ---
\begin{frame}[allowframebreaks, label=q10]{Q10: Các biến thể và mở rộng (Extensions)}
    \footnotesize
    \begin{block}{Câu hỏi}
        Có phiên bản mở rộng nào của Sign Test không? Khi nào dùng?
    \end{block}
    
    \vspace{0.2cm}
    \textbf{Các "người anh em" của Sign Test:}
    
    \begin{enumerate}\itemsep0pt
        \item \textbf{McNemar Test:}
        \begin{itemize}\itemsep0pt
            \item Dùng cho dữ liệu \textbf{Nhị phân (Binary)}: 0/1, Đậu/Trượt, Bệnh/Khỏi.
            \item Thiết kế bảng 2x2 contingency.
            \item Vẫn là so sánh Trước/Sau (Paired).
        \end{itemize}
        
        \item \textbf{Friedman Test:}
        \begin{itemize}\itemsep0pt
            \item Mở rộng cho \textbf{nhiều hơn 2 nhóm} (VD: Đo cân nặng tại T1, T2, T3).
            \item Dùng dữ liệu xếp hạng (Ranked data).
            \item Được coi là phiên bản phi tham số của ANOVA lặp lại (Repeated Measures ANOVA).
        \end{itemize}
        
        \item \textbf{Cochran's Q Test:}
        \begin{itemize}\itemsep0pt
            \item Giống Friedman nhưng dùng cho dữ liệu Nhị phân (Binary).
            \item Mở rộng của McNemar cho nhiều nhóm.
        \end{itemize}
    \end{enumerate}
    
    \vspace{0.2cm}
    \begin{block}{Sơ đồ quyết định (Decision Tree)}
        \begin{itemize}\itemsep0pt
            \item Data liên tục/thứ bậc, 2 nhóm $\to$ \structure{Sign Test} (hoặc Wilcoxon).
            \item Data nhị phân, 2 nhóm $\to$ \structure{McNemar Test}.
            \item Data liên tục/thứ bậc, >2 nhóm $\to$ \structure{Friedman Test}.
            \item Data nhị phân, >2 nhóm $\to$ \structure{Cochran's Q}.
        \end{itemize}
    \end{block}
    \vfill \hfill \hyperlink{qna_menu}{\beamerbutton{Quay lại Menu}}
\end{frame}